%!TeX root=../tese.tex
%("dica" para o editor de texto: este arquivo é parte de um documento maior)
% para saber mais: https://tex.stackexchange.com/q/78101/183146

% O resumo é obrigatório, em português e inglês. Este comando também gera
% automaticamente a referência para o próprio documento, conforme as normas
% sugeridas da USP

% Elemento obrigatório, constituído de uma sequência de frases concisas e
% objetivas, em forma de texto.  Deve apresentar os objetivos, métodos empregados,
% resultados e conclusões.  O resumo deve ser redigido em parágrafo único, conter
% no máximo 500 palavras e ser seguido dos termos representativos do conteúdo do
% trabalho (palavras-chave). Deve ser precedido da referência do documento.

\begin{resumo}{port}
Este trabalho possui dois objetivos, servir como um guia inicial de estudos sobre redes neurais no contexto de aprendizado de máquina, apresentando uma implementação didática do algoritmo de treinamento da rede \emph{Perceptron}, e testar se redes neurais possuem o potencial para prever séries temporais financeiras com um desempenho similar ou superior em comparação aos modelos paramétricos tradicionalmente usados, como o \eng{ARIMA}. Após uma série de comparações dos modelos, conclui-se que o resultado é dependente da quantidade de dados disponíveis para o treinamento, de forma que poucos dados disponíveis favorecem os modelos paramétricos, e quanto mais dados mais potencial há nas redes neurais, tanto em maior ou igual qualidade das previsões quanto em menor tempo relativo de processamento.
\end{resumo}


% Versão em inglês

\begin{resumo}{eng}
This work has two objectives, to serve as an initial guide for studies on neural networks in the context of machine learning, presenting a didactic implementation of the \emph{Perceptron} network training algorithm, and to test whether neural networks have the potential to predict financial time series with similar or better performance compared to parametric models traditionally used, such as \eng{ARIMA}. After a set of comparisons of the models, it is concluded that the result is dependent on the amount of data available for training, so that few data available favor the parametric models, and the more data the more potential there is in neural networks, both in greater or equal quality of forecasts as in less relative processing time.
\end{resumo}
