%!TeX root=../tese.tex
%(dica para o editor de texto: este arquivo é parte de um documento maior)
% para saber mais: https://tex.stackexchange.com/q/78101/183146

% https://otexts.com/fpp2/index.html
% https://www.tensorflow.org/tutorials/structured_data/time_series

\chapter{Séries temporais}
\label{cap:series}

Neste capítulo são apresentados os conceitos básicos de séries temporais, com destaque para séries históricas de conversões de moedas estrangeiras. Tais séries estão disponíveis para \eng{download} em diversos sites, incluindo o site do Banco Central do Brasil\footnote{\url{https://www.bcb.gov.br/}}. Também são discutidos brevemente os métodos tradicionais de análise das séries como a \emph{transformada de Fourier} e de previsões de valores futuros da série, como o \emph{SARIMA}.

A análise e predição de fatores de câmbios de moedas estrangeiras, preços de \emph{commodities}, ou outros dados econômicos, constituem uma área essencial da economia e que exige a avaliação de um número enorme de fatores, muitos dos quais de características humanas e portanto imprevisíveis em sua exatidão, até mesmo por se tratar de um fenômeno \defi{estocástico}, sujeito a incertezas.

De acordo com Pedro A. Morettin e Clélia M C. Toloi \citep{morettin}, uma série temporal é qualquer conjunto de observações ordenadas no tempo. Exemplos são valores diários de poluição de uma cidade, valores mensais de temperatura, índices diários da bolsa de valores, número médio anual de manchas solares e registro de marés em portos e estuários.

As séries temporais podem ser \emph{contínuas} em função do tempo, como é o ltimo exemplo do parágrafo anterior, ou \emph{discretas} como são todos os outros exemplos, ou seja os valores são tomados a intervalos regulares. Na prática, para o uso em modelos, segundo explica Morettin e Toloi \citep{morettin}, as séries contínuas devem ser discretizadas em intervalos, uma vez que é esse tipo de dado que poderá ser processado num computador.

Há dois tipos básicos de modelos que lidam com séries temporais, os modelos \defi{paramétricos} e os \defi{não-paramétricos}. Este capítulo explica a diferença e o exemplo do uso de ambos os tipos de modelos. Em primeiro lugar uma introdução ao conceito de processos estocásticos é necessária. Na última seção desse capítulo, uma estrutura de uma rede neural recorrente, implementada com a API Keras, é proposta como um modelo \defi{semiparamétrico} para a previsão de séries temporais financeiras.

\section{Processos estocásticos}

\section{Modelos não-paramétricos: transformada de Fourier e a função de autocorrelação}

A transformada discreta de Fourier é uma função $2\pi$-periódica e definida em: \[F:(x_j)_{j \in \mathbb{N}} \rightarrow (z_j)_{j \in \mathbb{N}} \] onde $ x_j \in [0, 2\pi]$ e $ z_j \in C $, com $C \subset \mathbb{R}$ ou $C \subset \mathbb{C}$. Ou seja, é tabelada em $2N$ pontos igualmente espaçados (no caso das oscilações usamos os preços de cada dia, portanto levando a valores reais, e o intervalo de dias é o domínio), no intervalo $[0, 2\pi]$ (basta normalizar o intervalo de dias nesse intervalo, ou seja, denotando $x_j = j\pi/N $, com $j = 0, 1, \dots, 2N-1$). A função pode ser definida para um conjunto de pontos que podem ser complexos ou mesmo reais. Faz isso associando aos valores $F(x_j)$ os coeficientes de Fourier $(c_k)$, dessa forma:
\begin{equation}\label{tfd_1}
F(x_j) = \sum_{k=0}^{2N-1} c_k e^{ikx_j}\ ,\ j=0, 1, \dots, 2N-1
\end{equation}
\begin{equation}\label{tfd_2}
c_k = \frac{1}{2N} \sum_{j=0}^{2N-1} F(x_j) e^{-ikx_j}\ ,\ k=0, 1, \dots, 2N-1
\end{equation}

\section{Modelos paramétricos autoregressivos e de médias móveis}

\section{Um modelo semiparamétrico construído com uma rede neural}