%!TeX root=../tese.tex
%(dica para o editor de texto: este arquivo é parte de um documento maior)
% para saber mais: https://tex.stackexchange.com/q/78101/183146

% https://otexts.com/fpp2/index.html
% https://www.tensorflow.org/tutorials/structured_data/time_series

\chapter{Séries temporais}
\label{cap:series}

Neste capítulo são apresentados os conceitos básicos de séries temporais. Também são discutidos brevemente os modelos tradicionais de análise e descrição das séries, ou de previsões de valores futuros, de acordo com o objetivo do estudo. Tais modelos são, por exemplo, baseados em médias móveis, tendências e sazonalidades presentes nos dados.

De acordo com Pedro A. Morettin e Clélia M C. Toloi \citep{morettin}, uma série temporal é qualquer conjunto de observações ordenadas no tempo. São exemplos: valores diários de poluição de uma cidade, valores mensais de temperatura, índices diários da bolsa de valores, número médio anual de manchas solares e registro de marés em portos e estuários.

A análise e predição de cotações de moedas estrangeiras, índices de ações, entre outros dados econômicos, constituem uma área essencial da economia e que exige a avaliação de um número enorme de fatores, muitos dos quais de características humanas e portanto imprevisíveis em sua exatidão, sendo descritos como exemplos de fenômenos \emph{estocásticos}, isto é, probabilísticos.

As séries temporais podem ser \defi{contínuas} em função do tempo, como é o exemplo de registros de marés, ou \defi{discretas} como são todos os outros exemplos, ou seja, os valores são tomados a $N$ intervalos regulares de um período $T$ considerado, tal que $N = T/\Delta t$. Na prática, para o uso em modelos, segundo explica Morettin e Toloi \citep{morettin}, as séries contínuas devem ser discretizadas em intervalos, uma vez que é esse tipo de dado que poderá ser processado num computador.

Pode-se classificar a análise das séries temporais de acordo com seu objetivo de estudo. Morettin e Toloi \citep{morettin} listam os alguns objetivos principais:

\begin{itemize}
	\item{Investigar o mecanismo gerador da série temporal, procurando descrevê-la a partir de uma função teórica.}
	\item{Fazer previsões de valores futuros da série, seja tanto a curto quanto a longo prazo.}
	\item{Descrição da série, em termos de tendências, variações sazonais, ou então análises descritivas por meio de histogramas, médias móveis, etc.}
	\item{Procurar por periodicidades relevantes nos dados, quando não fazemos suposições de periodicidades comuns como semanais, mensais ou anuais, por exemplo.}
\end{itemize}

Neste contexto, Morettin e Toloi \citep{morettin} definem que um modelo é uma descrição probabilística de uma série temporal, cabendo ao cientista de dados decidir a melhor utilização desse modelo segundo seus objetivos. 

Além disso eles afirmam que qualquer tarefa de previsão, sendo este o objetivo, será baseada em algum procedimento computacional que calcula uma estimativa do futuro baseada na otimização de uma função de perda calculada sobre combinações lineares de valores do passado. É a união de um modelo probabilístico com a otimização de uma função de perda que define um \defi{método de previsão}.

Há dois tipos básicos de modelos que lidam com séries temporais, de acordo com Morettin e Toloi \citep{morettin}, que são os modelos \defi{paramétricos}, onde a análise é feita no \emph{domínio temporal} com suposições e parâmetros a serem estimados, e os modelos \defi{não-paramétricos}, em que a análise é feita no \emph{domínio das frequências} com um enfoque mais descritivo e sem muitas suposições.

Dentre os modelos paramétricos, destacam-se os modelos \emph{ARIMA} (\emph{autorregressivos integrados de médias móveis}), disponíveis como bibliotecas das linguagens \eng{Python} e \eng{R}, enquanto que entre os modelos não-paramétricos destaca-se a \emph{análise espectral}, também denominada de \emph{análise de Fourier}, já que as ferramentas utilizadas são as transformadas de Fourier e suas variações.

Na última seção desse capítulo, uma estrutura de uma rede neural recorrente, implementada com a API Keras, é proposta como um modelo \defi{semiparamétrico}, ou seja, estimando parâmetros inerentes às redes neurais, mas não utilizando parâmetros ou suposições específicas sobre os dados, para a realização de previsões de valores futuros de séries temporais financeiras, com enfoque nas séries históricas de cotações de moedas estrangeiras.\footnote{Tais séries estão disponíveis para \eng{download} no site do Banco Central do Brasil: \url{https://www.bcb.gov.br/}}.

\section{Processos estocásticos}

De acordo com Morettin e Toloi \citep{morettin}, um \defi{processo estocástico} é uma familia $Z = \{ Z(t), t \in \mathcal{T} \}$, onde o conjunto $\mathcal{T}$ é normalmente tomado como $\mathcal{T} = \mathbb{Z}$ ou $\mathcal{T} = \mathbb{R}$, tal que, para cada $t \in \mathcal{T}$, $Z(t)$ é uma variável aleatória (v.a.). 

Portanto, um processo estocástico é uma família de variáveis aleatórias reais $Z(t),\; t \in \mathcal{T}$, definidas num mesmo espaço de probabilidades $(\Omega, \mathcal{A}, \mathcal{P})$, e portanto $Z(t)$ é uma função de dois argumentos, isto é, $Z(t, \omega),\; t \in \mathcal{T},\; \omega \in \Omega$.

Para cada $t \in \mathcal{T}$ fixado, $Z(t, \omega)$ será uma v.a. com uma distribuição de probabilidades. Pode haver uma função de densidade de probabilidade diferente para cada $t \in \mathcal{T}$, mas normalmente assume-se que é a mesma, conforme anotado por Morettin e Toloi \citep{morettin}.

Por outro lado, para cada $\omega_i \in \Omega$ fixado, denotamos $Z(t, \omega_i)$ por $Z^{(i)}(t)$ como uma função de $t$, denominada de \defi{trajetória} do processo, ou simplesmente de \emph{série temporal}. Isto define uma série temporal como uma trajetória ou realização de um processo estocástico, o que pode ser melhor ilustrado pela Figura \ref{fig:trajetorias}, abaixo.

\begin{figure}[htb]
\centering
\includegraphics[width=10cm]{figuras/trajetorias}
\caption{Processo estocástico como uma família de trajetórias, isto é, de séries temporais.\footnote{Extraido de Morettin e Castro Toloi, 2019, pág 27.}}
\label{fig:trajetorias}
\end{figure}

Estaremos interessados em processos univariados tanto em $\mathcal{T}$ quanto em $\Omega$, isto é, séries temporais de apenas um argumento temporal, e com um evento $\omega \in \Omega$ fixado e portanto omitido, o que simplifica a notação de $\{Z^{(i)}(t), t \in \mathcal{T}\}$ para $\{Z(t), t \in \mathcal{T}\}$, o que definem as \emph{séries temporais univariadas}, que descrevem como uma v.a. real evolui no domínio temporal $\mathcal{T}$.

Além disso, para as séries e modelos aqui tratados iremos restringir $\mathcal{T} = \mathbb{Z}$, assim omitiremos a definição de um domínio geral e fixaremos a notação como sendo $\{Z(t), t \in \mathbb{Z}\}$ ou ainda $\{Z_t, t \in \mathbb{Z}\}$, para denotar as \emph{séries temporais discretas univariadas}, que além de discretas são equiespaçadas no tempo. A partir de agora, serão chamadas simplesmente de \defi{séries temporais}, pois são os únicos processos estocásticos de interesse neste trabalho.

\subsection{Definições}

Seguem algumas definições que nos levarão a classes específicas de processos estocásticos, com as quais lidaremos daqui em frente. Sejam $t_1, \ldots, t_n$ elementos quaisquer de $\mathcal{T}$, daí se conhecermos as \defi{distribuições finito-dimensionais} de $Z$ dadas por:

\begin{equation}\label{series:2.1}
F(z_1, \ldots, z_n; t_1, \ldots, t_n) = P\{ Z_{t_1} \leq z_1, \ldots, Z_{t_n} \leq z_n \}
\end{equation}

Teremos então que o processo estocástico $Z = \{ Z_t, t \in \mathcal{T} \}$ estará especificado, para todo $n \geq 1$. Tais funções de distribuição devem, de acordo com Morettin e Toloi \citep{morettin} satisfazer as condições:

	(i) (Simetria) Para qualquer permutação $j_1, \ldots, j_n$ dos índices $1, \dots, n$:
\[ F(z_{j_1}, \ldots, z_{j_n}; t_{j_1}, \ldots, t_{j_n}) = F(z_1, \ldots, z_n; t_1, \ldots, t_n) \]

	(ii) (Compatibilidade) Para $m < n$:
\[ \lim_{\mathlarger{z_{m+1} \to \infty, \ldots, z_n \to \infty}} F(z_1, \ldots, z_m, z_{m+1}, \ldots, z_n; t_1, \ldots, t_n) = F(z_1, \ldots, z_m; t_1, \ldots, t_m) \]

Segundo Morettin e Toloi \citep{morettin}, pode-se demonstrar que qualquer conjunto de funções de distribuição da forma \ref{series:2.1} satisfazendo as duas condições acima define um processo estocástico $Z$ sobre $\mathcal{T}$.

Em termos práticos, não se conhecem as funções de distribuição finito-dimensionais de um processo $Z$ sobre $\mathcal{T}$. Assim a abordagem mais utilizada, conforme Morettin e Toloi \citep{morettin} é tentar determinar os momentos, principalmente os de primeira e segunda ordem, das v.a. $Z_{t_1}, \ldots, Z_{t_n}$. 

O momento de primeira ordem, isto é, a \defi{média} de $Z$ é definida por: 

\begin{equation}\label{series:2.5}
\mu(1; t) = \mu(t) = \E\{Z(t)\} = \int_{-\infty}^{\infty} z f(z;t)dz, t \in \mathcal{T}
\end{equation}

Define-se, a partir dos momentos de primeira ordem, a \defi{função de autocovariância} (facv) de $Z$:

\begin{equation}\label{series:2.6}
\gamma(t_1, t_2) = \mu(1,1; t_1,t_2) - \mu(1; t_1)\mu(1; t_2) = \Cov\{Z(t_1),Z(t_2)\}
\end{equation}

Particularmente, quando $t = t_1 = t_2$, define-se a função \defi{variância} de $Z$, configurando um momento de segunda ordem, por:

\begin{equation}\label{series:2.7}
\gamma(t, t) = \Var\{Z(t)\} = \E\{Z^2(t)\} - \E^2\{Z(t)\}
\end{equation}

\subsection{Processos estacionários}

Um processo $Z$ é dito \defi{estacionário} se suas características para qualquer tempo não dependem da escolha da origem do domínio temporal, isto é, as características de $Z(t + \tau)$ para todo $\tau$, são as mesmas de $Z(t)$. Morettin e Toloi \citep{morettin} nomeia o parâmetro $\tau$ de ``\defi{lag}''\footnote{jargão em inglês que em português pode significar latência ou atraso.}, e dá como exemplo de processo estacionário as medidas de vibrações de um avião em regime estável de vôo.

Formalmente, um processo estocástico $Z = \{Z(t), t \in \mathcal{T}\}$ é \defi{fracamente estacionário} ou estacionário de segunda ordem, se e somente se:

	(i) $\E\{Z(t)\} = \mu(t) = \mu, \;\;\; \forall ~ t \in \mathcal{T};$

	(ii) $\E\{Z^2(t)\} < \infty, \;\;\; \forall ~ t \in \mathcal{T};$

	(iii) $\gamma(t_1, t_2) = \Cov\{Z(t_1), Z(t_2)\}$ é uma função de $|t_1 - t_2|, \;\;\; \forall~ t_1, t_2 \in \mathcal{T}.$

Dessa forma, podemos dizer que processos estacionários de segunda ordem desenvolvem-se em torno de uma média constante, ou seja, ao redor de uma mesma tendência ou reta. É possível citar dois tipos de não-estacionaridade. 

Existem os processos \emph{homogêneos}, que apresentam uma estacionariedade inicial mas que depois sofrem uma mudança de tendência e então tornam-se estacionárias novamente, mas não necessariamente ao redor da mesma média inicial, e que segundo Morettin e Toloi \citep{morettin} podem se tornar estacionários se tomarmos diferenças sucessivas da série original.

Tomar diferenças de uma série $Z(t)$ corresponde a criar uma nova série a partir da original. A primeira diferença de $Z(t)$ é definida por:

\begin{equation}\label{series:2.12}
\Delta Z(t) = Z(t) - Z(t - 1)
\end{equation}

Recursivamente, a partir dessa primeira definição, podemos escrever a n-ésima diferença de $Z(t)$ como sendo:

\begin{equation}\label{series:2.13}
\Delta^n Z(t) = \Delta[\Delta^{n-1} Z(t)]
\end{equation}

Adicionalmente, aplicar a transformação não-linear $\log Z(t)$ também pode ser útil para transformar uma série não-estacionária em uma série estacionária. De acordo com Morettin e Toloi \citep{morettin}, a transformação logarítmica é muito usada em séries econômicas e será apropriada se a variância da série for proporcional à média.

Para exemplificar esses conceitos, considere a Figura \ref{fig:exe_estac}. Temos à esquerda uma série temporal que é uma realização de um processo não-estacionário homogêneo, a saber, índices mensais da bolsa de valores Ibovespa. À direita, foi aplicado o logaritmo da primeira diferença da série, o que gerou uma série estacionária.

\begin{figure}[htb]
\centering
\includegraphics[width=14cm]{figuras/exemplo_estac}
\caption{Esquerda: Índices mensais do Ibovespa. Direita: Log-diferença do Ibovespa.\footnote{Extraido de Morettin e Castro Toloi, 2019, pág 7.}}
\label{fig:exe_estac}
\end{figure}

O segundo tipo de não-estacionaridade é chamada de \emph{explosiva}. Um exemplo de um processo não-estacionário explosivo é uma série temporal que descreve o crescimento de uma população de bactérias. Não tratamos de processos deste tipo neste trabalho.

\subsection{Função de autocorrelação}

Seja $\{ X_t, t \in \mathbb{Z} \}$ um processo estacionário real com tempo discreto, de média $\mu = 0$ e facv $\gamma_\tau = \E\{ X_t, X_{t+\tau} \}$. Sob essas condições, Morettin e Toloi \citep{morettin} demonstram que a facv $\gamma_\tau$ satisfaz as propriedades:

	(i) $\gamma_0 > 0$,

	(ii) $\gamma_{-\tau} = \gamma_\tau$,

	(iii) $|\gamma_\tau| \leq \gamma_0$,

	(iv) $\sum_{j=1}^n \sum_{k=1}^n a_j a_k \gamma_{\tau_j - \tau_k} \geq 0 \;\;\; \forall a_1, \ldots, a_n \in \mathbb{R}$ e $\tau_1, \ldots, \tau_n \in \mathbb{Z}$,

	(v) $\lim_{|\tau| \to \infty}  \gamma_\tau = 0$.

Define-se a \defi{função de autocorrelação} (fac) de um processo estocástico por:

\begin{equation}\label{series:2.14}
\rho_\tau = \frac{\gamma_\tau}{\gamma_0}
\end{equation}

A fac de um processo estacionário possui todas as propriedades da facv acima listadas e, em particular, $\rho_0 = 1$. O mais interessante é a ressalva de Morettin e Toloi \citep{morettin} de que a recíproca é verdadeira, isto é, dado um processo cuja fac ou facv possuem essas propriedades, então ele é estacionário. Dessa forma, temos um arcabouço teórico que nos permite investigar a existência da propriedade estacionária de uma série temporal dada.

\subsection{Exemplos de processos estocásticos}

O exemplo mais simples é o de uma \emph{sequência aleatória}. Uma sequência de v.a. definidas num mesmo espaço amostral $\Omega$ dada por $\{ X_n , n = 1, 2, \ldots \}$ é um processo estocástico com parâmetro discreto, ou seja, $\mathcal{T} = \{ 1, 2, \ldots \}$. Para todo $n \geq 1$ temos, em geral, que:

\[ P\{X_1 = a_1, \ldots, X_n = a_n\} = P\{X_1 = a_1\}\times P\{X_2 = a_2|X_1 = a_1\} \]
\[ \times\ldots\times P\{X_n = a_n|X_1 = a_1, \ldots, X_{n-1} = a_{n-1}\}\]

Se simplificarmos esse caso geral para o caso de uma sequência $\{ X_n , n \geq 1 \}$ de v.a. \emph{mutuamente independentes} então:

\[ P\{X_1 = a_1, \ldots, X_n = a_n\} = P\{X_1 = a_1\}\times\ldots\times P\{X_n = a_n\} \]

E se, além disso, todas as v.a. dessa sequência tiverem a mesma distribuição de probabilidades, elas serão portanto independentes e identicamente distribuidas (i.i.d.), o que configura $X_n = \{ X_n , n \geq 1 \}$, uma sequência de v.a. i.i.d., como um processo estocástico estacionário.

Definindo $\E\{X_n\} = \mu$, $\Var\{X_n\} = \sigma^2$, para todo $n \geq 1$, teremos que a facv de $X_n$ será dada por:

\begin{equation}\label{series:2.19}
\gamma_\tau = \Cov\{X_n, X_{n+\tau}\} = \left\{\begin{array}{lr} \sigma^2, & \text{ se } \tau = 0 \\ 0, & \text{ se } \tau \neq 0 \end{array}\right.
\end{equation}

E, a fac de $X_n$, será tal que:

\begin{equation}\label{series:2.20}
\rho_\tau = \left\{\begin{array}{lr} 1, & \text{ se } \tau = 0 \\ 0, & \text{ se } \tau \neq 0 \end{array}\right.
\end{equation}

Um segundo exemplo de processo estocástico, e muito mais útil para os estudos de séries temporais, é o de \defi{ruído branco}. Morettin e Toloi \citep{morettin} definem que a sequência $\{\epsilon_t,\; t \in \mathbb{Z}\}$ é um \emph{ruído branco discreto} se as v.a. $\epsilon_t$ não são correlacionadas, ou seja, $\Cov\{\epsilon_t, \epsilon_s\} = 0, t \neq s$.

Esse processo será estacionário se $\E\{\epsilon_t\} = \mu_\epsilon$ e $\Var\{\epsilon-t\} = \sigma_\epsilon^2$, para todo $t \in \mathbb{Z}$. Dessa forma, as facv e fac de um ruído branco serão dadas, respectivamente e analogamente, por \ref{series:2.19} e \ref{series:2.20}. Tipicamente representa-se o gráfico de uma função de autocorrelação conforme o exemplo do fac de um ruído branco, dado na Figura \ref{fig:fac_ruido}.

\begin{figure}[htb]
\centering
\includegraphics[width=10cm]{figuras/fac_ruido}
\caption{Função de autocorrelação (fac) de um ruído branco.\footnote{Extraido de Morettin e Castro Toloi, 2019, pág 35.}}
\label{fig:fac_ruido}
\end{figure}

Normalmente, é dito por Morettin e Toloi \citep{morettin} que, sem perda de generalidade, podemos assumir que a média de um ruído branco é zero. E, assim, escrevemos:

\[ \epsilon_t \sim RB (0, \sigma_\epsilon^2). \]

Se, as v.a. de $\epsilon_t$ forem independentes, então é um resultado de probabilidade conhecido e citado por Morettin e Toloi \citep{morettin} que serão também não correlacionados. Um ruído branco, como definido acima, e com a propriedade adicional da independência, é um terceiro exemplo de um processo estocástico, chamado de \emph{processo puramente aleatório}. Nesse caso, escrevemos:

\[ \epsilon_t \sim i.i.d. (0, \sigma_\epsilon^2). \]

Outros exemplos de processos estocásticos que podemos citar, menos formais e mais físicos, são os \emph{passeios aleatórios}, que fazem por exemplo, a modelagem de um sistema de partículas livres, como as moléculas de um gás ideal. E, outro exemplo, análogo, é o do \emph{movimento browniano}, que modela, por exemplo, como partículas de poeira movimentam-se num fluído visto como um conjunto de muitas moléculas ligadas entre si, como a água no estado líquido.

\section{Modelos para séries temporais}

Já foi citado acima que existem os modelos parâmétricos, que lidam com as séries no domínio temporal, e os modelos não-paramétricos que lidam com o \defi{espectro} da série, isto é, o conjunto das frequências da série, onde as componentes desse espaço das frequências são ortogonais entre si, o que garante a não correlação entre as frequências, diferindo dos valores sob o domínio temporal que quase sempre são correlacionados entre si, conforme explicado por Morettin e Toloi \citep{morettin}.

Se temos uma série temporal discreta $Z = \{ Z_t, t \in \mathbb{Z} \}$, que assumimos ser um processo estacionário, e assumindo também que as autocovariâncias $\gamma_\tau$ são tais que $\sum_{\tau=-\infty}^{\infty} |\gamma_\tau| < \infty$, então o espectro de $Z$, isto é, a \defi{transformada de Fourier} de $Z$ será:

\begin{equation}\label{series:2.33}
f(\omega) = \frac{1}{2\pi} \sum_{\tau=-\infty}^{\infty} \gamma_\tau e^{-i\omega\tau}, \;\;\; -\pi\leq\omega\leq\pi.
\end{equation}

Ora, isto redefine a função de autocovariância (facv), como sendo a \emph{anti-transformada} de $Z$, pois:

\begin{equation}\label{series:2.34}
\gamma_\tau = \int_{-\pi}^{\pi} e^{i\omega\tau} f(\omega) d\omega, \;\;\; \tau \in \mathbb{Z}.
\end{equation}

Embora úteis para a construção de modelos de engenharia e física, Morettin e Toloi \citep{morettin} ressaltam que usar o espectro ou a facv para modelar uma série temporal é mais útil no contexto de processos industriais, e o que o uso mais prático em outros contextos, como o de séries financeiras é o papel que desempenham nos modelos paramétricos como o ARIMA.

Isto advém do fato de que os modelos paramétricos tem comportamentos esperados, com valores de facv, em alguns casos, muito bem definidos e que podem ser testados e usados como uma validação ao modelo que se supõe ter o mesmo comportamento que a série temporal de estudo.

%\section{Um modelo de série temporal construído com uma rede neural}