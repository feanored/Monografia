%!TeX root=../tese.tex
%(dica para o editor de texto: este arquivo é parte de um documento maior)
% para saber mais: https://tex.stackexchange.com/q/78101/183146

\chapter{Perceptron multi-camadas}
\label{cap:perceptron}

Neste capítulo é descrita a implementação e funcionamento de uma versão do algoritmo \eng{perceptron}, feito a partir de um núcleo básico disponibilizado no livro de Kopec \citep{classic}, e a partir do qual foram feitas modificações e criação de novos métodos de treinamento, de validação e de avaliação do treinamento.

O perceptron aqui implementado pode ser usado tanto para tarefas de classificações, quanto para tarefas de regressões de dados, e neste capítulo são apresentados exemplos de ambos os usos, embora o foco deste trabalho seja na versão de classificação.

\section{Derivação matemática do algoritmo de retropropagação}

Para a aprendizagem supervisionada foi utilizado o algoritmo de retropropagação (\eng{retropropagation}), que consiste na minimização de uma função de custos, a partir do gradiente, ou seja, da derivada desta função de custos, neste caso o erro quadrático médio, conforme foi definido no capítulo anterior.

De acordo com Kopec \citep{classic}, o perceptron consiste de uma rede cujo sinal, ou seja, os dados, se propagam em uma só direção, da camada de entrada para a camada de saída, passando pelas camadas ocultas uma a uma, e por isso o nome de rede \eng{feedforward} ao perceptron. Por sua vez, o erro que determinamos na camada final propaga-se no caminho inverso, sendo distribuídas correções da saída para a entrada, afetando aqueles neurônios que foram mais responsáveis pelo erro total. Por isso o nome de retropropagação.

Estendendo as definições já usadas no capítulo anterior, segue a derivação matemática do algoritmo de retropropagação. Como ficará claro mais à frente, podemos derivar as contas para apenas um neurônio por camada sem perda de generalidade. Dessa forma, se temos uma rede com $L$ camadas, o erro quadrático para um neurônio da camada de saída (a camada $L$) será:
\[
C_0 = (a^{(L)} - y)^2
\]
onde $y$ é a saída esperada, e $a^{(L)}$ é a saída de um neurônio da camada de saída.

Temos que $C_0$ é uma função de $a^{(L)}$, uma vez que $y$ é um valor fixo conhecido. Por sua vez, temos que de modo geral a saída de um neurônio é uma função do tipo:
\[
a^{(L)} = \sigma(w^{(L)}~a^{(L-1)} + b^{(L)})
\]
onde escrevemos $a^{(L-1)}$ é a saída do neurônio da camada anterior, $w^{(L)}$ é o \defi{peso} atribuido a essa saída, o que seria o parâmetro angular $A$ na Figura \ref{fig:neuronio}, e $b^{(L)}$ é o chamado \defi{viés} desse neurônio, análogo ao parâmetro linear de uma reta. Por fim temos a \defi{função de ativação} que escrevemos como $\sigma$ que é aplicada à essa equação linear.

Nota-se que internamente à função de ativação, um neurônio se comporta como uma transformação linear dos neurônios da camada anterior. Caso tivéssemos $n$ neurônios na camada anterior à de saída, teríamos então $n$ pesos, denotados com índice $i$ dessa forma: $\{ w_i^{(L)} \}_{i=1}^n$. Cabe assim à função de ativação, dar o comportamento não-linear à rede perceptron.

Como o objetivo é minimizar $C_0$, temos que calcular a influência dos pesos e dos viéses nesse custo. Já sabemos que isso será obtido com o gradiente, isto é, a derivada dessa função em relação a esses parâmetros que, são os únicos que podemos otimizar. De forma mais clara, temos que no início do treinamento da rede, atribuímos valores aleatórios aos pesos e aos viéses, e então executamos o \eng{feedforward}, de forma que a rede irá calcular sequencialmente os valores de saída em todas as suas camadas, obtidos a partir dos dados de entrada, que serão fixos, e desses parâmetros inicialmente aleatórios. A partir daí, poderemos otimizar esses parâmetros, exatamente da forma que estamos construindo.

O cálculo dessas derivadas é feito segundo a regra da cadeia, e adicionalmente iremos denotar a transformação linear interna à função de ativação por $z^{(L)} = w^{(L)}~a^{(L-1)} + b^{(L)}$, de forma que $a^{(L)} = \sigma(z^{(L)})$. Assim, ficamos com as derivadas para a camada de saída:

\begin{equation}\label{retro:1}
\frac{\del C_0}{\del w^{(L)}} = \frac{\del z^{(L)}}{\del w^{(L)}} \frac{\del a^{(L)}}{\del z^{(L)}} \frac{\del C_0}{\del a^{(L)}}
\end{equation}

\begin{equation}\label{retro:2}
\frac{\del C_0}{\del b^{(L)}} = \frac{\del z^{(L)}}{\del b^{(L)}} \frac{\del a^{(L)}}{\del z^{(L)}} \frac{\del C_0}{\del a^{(L)}}
\end{equation}

Para a camada de saída, podemos calcular diretamente cada termo dessas derivadas:

\begin{equation}\label{retro:4}
\frac{\del C_0}{\del a^{(L)}} = 2(a^{(L)} - y) ~\propto~ (a^{(L)} - y)
\end{equation}

\begin{equation}\label{retro:5}
\frac{\del a^{(L)}}{\del z^{(L)}} = \sigma^{'}(z^{(L)})
\end{equation}

\begin{equation}\label{retro:6}
\frac{\del z^{(L)}}{\del w^{(L)}} = a^{(L-1)}
\end{equation}

\begin{equation}\label{retro:7}
\frac{\del z^{(L)}}{\del b^{(L)}} = 1
\end{equation}

O que resulta, fazendo todas as substituições, em:

\begin{equation}\label{retro:10}
\frac{\del C_0}{\del w^{(L)}} = a^{(L-1)}~ \sigma^{'}(z^{(L)})~ (a^{(L)} - y)
\end{equation}

\begin{equation}\label{retro:11}
\frac{\del C_0}{\del b^{(L)}} = \sigma^{'}(z^{(L)})~ (a^{(L)} - y)
\end{equation}

Na equação \ref{retro:4} ocultamos o termo constante $2$ sob um símbolo de proporção, que a seguir iremos também ocultar, uma vez que usaremos o algoritmo do gradiente descendente, e assim, em seu lugar, e na verdade, todas as derivadas aqui mostradas serão multiplicadas pelo termo $\eta$, a \defi{taxa de aprendizagem}, conforme explicado no capítulo anterior. 

Analogamente, podemos pensar numa forma de fazer esses cálculos para as camadas ocultas. A princípio, podemos calcular:

\begin{equation}\label{retro:3}
\frac{\del C_0}{\del a^{(L-1)}} = \frac{\del z^{(L)}}{\del a^{(L-1)}} \frac{\del a^{(L)}}{\del z^{(L)}} \frac{\del C_0}{\del a^{(L)}}
\end{equation}

Usando o fato de que:

\begin{equation}\label{retro:8}
\frac{\del z^{(L)}}{\del a^{(L-1)}} = w^{(L)}
\end{equation}

Agora, seja a $i$-ésima camada oculta tal que $1 < i < L$, se observarmos a equação \ref{retro:3}, e fizermos $i = L-1$, usando a equação \ref{retro:8}, ficamos com:

\begin{equation}\label{retro:9}
\frac{\del C_0}{\del a^{(i)}} = w^{(i+1)} \frac{\del a^{(i+1)}}{\del z^{(i+1)}} \frac{\del C_0}{\del a^{(i+1)}}
\end{equation}

Podemos observar que há um mesmo termo duplo que aparece tanto nas equações \ref{retro:1} e \ref{retro:2} quanto na equação \ref{retro:9} acima, de forma que apenas o índice da camada é diferente. Para simplificar podemos nomear esse termo de \emph{delta da camada $i$}:

\begin{equation}\label{retro:12}
\Delta^{(i)} = \dfrac{\del a^{(i)}}{\del z^{(i)}} \dfrac{\del C_0}{\del a^{(i)}}
\end{equation}

Simplificando todas as demais expressões usando essa definição, ficamos com:

\begin{equation}\label{retro:13}
\frac{\del C_0}{\del w^{(i)}} = a^{(i-1)} \Delta^{(i)}
\end{equation}

\begin{equation}\label{retro:14}
\frac{\del C_0}{\del b^{(i)}} = \Delta^{(i)}
\end{equation}

Como vemos, as derivadas que precisamos todas dependem desse termo $\Delta$, que por sua vez depende do cálculo do termo $\dfrac{\del C_0}{\del a^{(i)}}$ que será calculado de 2 formas distintas:

\[ \dfrac{\del C_0}{\del a^{(i)}} = w^{(i+1)}~ \Delta^{(i+1)}  \;\;\Rightarrow \]
\begin{equation}\label{retro:15}
\Delta^{(i)} = \sigma^{'}(z^{(i)})~ w^{(i+1)}~ \Delta^{(i+1)}
\end{equation}
para as camadas ocultas.


\[ \dfrac{\del C_0}{\del a^{(L)}} = (y - a^{(L)}) \;\;\Rightarrow \]
\begin{equation}\label{retro:16}
\Delta^{(L)} = \sigma^{'}(z^{(L)})~ (y - a^{(L)})
\end{equation}
para a camada de saída.

Percebe-se a natureza recursiva do algoritmo, onde o caso base é calculado na camada de saída, e que o cálculo vai propagando-se para as camadas ocultas, em direção à camada de entrada. Por essa mesma razão, pudemos derivar as contas para uma camada, e no fim elas estão prontas pra serem implementadas para qualquer número de camadas ocultas. 

Outro fato útil é que a expressão interna do neurônio é uma transformação linear, assim as contas podem ser facilmente ajustadas para o caso geral em que há $n_i$ neurônios em dada camada $i$ da rede, conforme já explicado, e que será detalhado diretamente nos trechos de código que serão mostrados a seguir na implementação propriamente dita.

\section{Implementação do algoritmo de retropropagação}

