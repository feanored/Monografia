%!TeX root=../tese.tex
%("dica" para o editor de texto: este arquivo é parte de um documento maior)
% para saber mais: https://tex.stackexchange.com/q/78101/183146

%% ------------------------------------------------------------------------- %%
\chapter{Introdução}
\label{cap:introducao}

%[Em geral a introdução é algo que procura ser atraente, motivacional e elementar.]

Algoritmos de aprendizado de máquina vem sendo utilizados para resolver um enorme gama de problemas computacionais de resolução numérica. Existem atualmente aplicações em praticamente todas as áreas do conhecimento humano, da agricultura à indústria e ao entretenimento. Mas o que é \defi{aprendizado de máquina}? Ou melhor dizendo, o que significa dizer que o computador, ou seja, a ``máquina'' está \emph{aprendendo}? 

Antes da definição propriamente dita, Aurélien Géron \citep{hands} nos dá uma ideia geral lembrando que uma das primeiras aplicações de sucesso de aprendizado de máquina foi o filtro de \eng{spam}, criado na década de 90. Uma das fases de seu desenvolvimento foi aquela em que os usuários marcavam que certos emails era \eng{spams} e outros não eram. Hoje em dia, raramente temos que marcar ou desmarcar emails, pois a maioria dos filtros já ``aprenderam'' a fazer seu trabalho de forma muito eficiente.

Existem muitas definições do que seja aprendizado de máquina, de modo geral uma área da ciência da computação mas que, especificamente no contexto de ciência de dados, Joel Grus \citep{data} define como a criação e o uso de modelos que são aprendidos a partir dos dados. O objetivo é usar dados existentes para desenvolver modelos que possamos usar para \emph{prever} possíveis saídas para dados novos. Exemplos, além do filtro de \eng{spams} podem ser: prever transações de crédito fradulentas, prever a chance de um cliente clicar em uma propaganda ou então prever qual time de futebol irá vencer o Campeonato Brasileiro.

Uma \emph{rede neural} é um exemplo de modelo preditivo de aprendizado de máquina. Tal modelo foi criado com inspiração no funcionamento do cérebro biológico, e que, apesar de terem sido as primeiras a serem criadas, conforme descrito por David Kopec \citep{classic}, vem ganhando nova importância na última década, graças ao avanço computacional, uma vez que exigem muito processamento, e também porque podem ser usadas para resolver problemas de aprendizagem dos mais variados tipos.

Atualmente existem vários tipos de redes neurais, porém este trabalho lida principalmente com aquele tipo que foi originalmente criado inspirado no cérebro, chamado de \defi{perceptron}, e que portanto tenta imitar o comportamento dos neurônios e suas conexões, aprendendo padrões a partir de dados existentes e tentando prever o comportamento de dados novos a partir do padrão aprendido.

[Ao final a uma breve descrição do conteúdo do texto.]