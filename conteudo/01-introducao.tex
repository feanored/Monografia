%!TeX root=../tese.tex
%("dica" para o editor de texto: este arquivo é parte de um documento maior)
% para saber mais: https://tex.stackexchange.com/q/78101/183146

%% ------------------------------------------------------------------------- %%
\chapter{Introdução}
\label{cap:introducao}

%[Em geral a introdução é algo que procura ser atraente, motivacional e elementar.]

De tempos pra cá, ler e ouvir falar de \defi{ciência de dados} tornou-se muito comum, tanto nos meios profissionais e científicos quanto na mídia. Existem atualmente aplicações em praticamente todas as áreas do conhecimento humano, da agricultura à indústria e ao entretenimento.

Uma busca rápida na \emph{Wikipedia} \citep{wiki} define ciência de dados como um conjunto de ferramentas que extrai informações ou previsões a partir de um grande volume de dados, que podem ser números, textos, áudio, vídeo, entre outros, para ajudar na tomada de decisões de negócios.

Apesar de não ser a única definição para o termo, Pedro A. Morettin e Julio M. Singer \citep{apostila} nos lembram que essa também é uma definição da estatística. Eles comparam o uso dos termos e apontam que o trabalho dos \emph{cientistas de dados} diferem dos \emph{estatísticos} apenas quando eles usam dados de natureza multimídia como áudio e vídeo, por exemplo. Mas que, uma vez que esses dados são processados e tornam-se números, as técnicas e conceitos utilizados pelos primeiros passam a ser basicamente os mesmos utilizados pelos segundos.

Na verdade, Morettin \& Singer \citep{apostila} citam que na década de 80 houve uma primeira tentativa de aplicar o rótulo \emph{ciência de dados}, (\emph{Data Science}), ao trabalho feito pelos estatísticos aplicados da época, como uma forma de dar-lhes mais visibilidade. Curiosamente, fato mencionado pelos autores, existem atualmente cursos específicos de ciência de dados em universidades ao redor do mundo, mas a maioria deles situada em institutos de áreas aplicadas como engenharia e economia, e raramente nos institutos de estatística propriamente ditos.

Para entender um pouco mais de seu escopo, David M. Blei e Padhraic Smyth \citep{blei} discutem ciência de dados sob as visões estatística, computacional e humana. Eles argumentam que é a combinação desses três componentes que formam a essência do que ela é e, assim como, do conhecimento que ela é capaz de produzir.

Em resumo, a estatística guia a coleta e análise dos dados. A computação cria algoritmos, técnicas de processamento e gerenciamento de memória eficazes para que sua execução seja efetiva. E o papel humano é o de avaliar quais tipos de dados, técnicas de análises, algoritmos e modelos são apropriados para responder ao problema em questão. Este é o papel do \emph{cientista de dados}.

Algoritmos de \defi{aprendizado de máquina} vem sendo utilizados em grande parte dos modelos de ciência de dados. Mas o que é aprendizado de máquina? Ou então, o que significa dizer que o computador, neste caso a ``máquina'', está \emph{aprendendo}?

Aurélien Géron \citep{hands} nos dá uma ideia geral lembrando que uma das primeiras aplicações de sucesso de aprendizado de máquina foi o filtro de \eng{spam}, criado na década de 90. Uma das fases de seu desenvolvimento foi aquela em que os usuários assinalavam que certos e-mails eram \eng{spams} e outros não eram. Hoje em dia, raramente temos que marcar ou desmarcar e-mails, pois a maioria dos filtros já ``aprenderam'' a fazer seu trabalho de forma muito eficiente, não temos mais nada a ``ensiná-lo''.

O conceito de aprendizado de máquina está intimamente ligado à ciência da computação. Porém, no contexto de ciência de dados, é definido por Joel Grus \citep{data} como a ``criação e o uso de modelos que são ajustados a partir dos dados''. Seu objetivo é usar dados existentes para desenvolver modelos que possamos usar para \emph{prever} possíveis respostas à consultas. Exemplos, além do filtro de \eng{spams} podem ser: detectar transações de crédito fraudulentas, calcular a chance de um cliente clicar em uma propaganda ou então prever qual time de futebol irá vencer o Campeonato Brasileiro.

Como ficará claro ao longo deste trabalho, o aprendizado consiste na utilização de dados já conhecidos para ajustar parâmetros de modelos. Uma vez ajustados os parâmetros, o algoritmo que descreve o modelo passa a ser usado para responder às consultas. Essa fase de ajuste de parâmetros é chamada de aprendizado ou treinamento. 

Uma \defi{rede neural} é um exemplo de modelo preditivo de aprendizado de máquina que foi criado com inspiração no funcionamento do cérebro biológico. David Kopec \citep{classic} descreve que apesar de terem sido as primeiras a serem criadas, elas vem ganhando nova importância na última década, graças ao avanço computacional, uma vez que exigem muito processamento, e também porque podem ser usadas para resolver problemas de aprendizagem dos mais variados tipos.

Atualmente existem vários tipos de redes neurais, porém este trabalho lida principalmente com aquele tipo que foi originalmente criado sob a inspiração do funcionamento do cérebro, chamado de \defi{perceptron}, e que portanto tenta imitar o comportamento dos neurônios e suas conexões, aprendendo padrões a partir de dados existentes e tentando prever o comportamento de dados novos a partir do padrão aprendido.

Uma visão geral da arquitetura do aprendizado de máquina, situando a posição das redes neurais e do algoritmo \eng{perceptron} em toda esta estrutura, assim como exemplos de aplicações em cada uma das suas ramificações, estão no Capítulo \ref{cap:redes}.

Neste trabalho é utilizada como base uma versão simples do algoritmo \eng{perceptron} feita e explicada por Kopec \citep{classic}, e a partir desta base, foram criados novos métodos de treinamento e de validação com algumas estratégias, como uma tentativa de automatizar e otimizar o processo de treinamento do algoritmo, que é comumente feito de forma heurística. Detalhes dessa implementação estão no Capítulo \ref{cap:perceptron}.

São apresentados os conceitos e usos das séries temporais de dados não-lineares no Capítulo \ref{cap:series}, assim como alguns exemplos de aplicações na área financeira. As técnicas tradicionais de análise e de previsão de séries temporais de dados serão brevemente apresentadas neste capítulo.

Para servir de validação e aplicação do algoritmo criado, no Capítulo \ref{cap:comparacao} serão feitas comparações de desempenho entre o \eng{perceptron} e os modelos tradicionais de previsões de séries temporais apresentados no capítulo anterior, como o \defi{SARIMA} (\eng{seasonal auto regressive integrated moving average}). Tais comparações usam como inspiração trabalhos similares como o de Alsmadi, Omar, Noah e Almarashdah \citep{4809024}.

Ao final concluo sobre os modelos de previsões aqui estudados e comparados, destacando a qualidade e eficiência dos métodos de redes neurais, tão bons e às vezes melhores do que os métodos tradicionais estatísticos.

