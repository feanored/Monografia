%!TeX root=../tese.tex
%("dica" para o editor de texto: este arquivo é parte de um documento maior)
% para saber mais: https://tex.stackexchange.com/q/78101/183146

%% ------------------------------------------------------------------------- %%
\chapter{Introdução}
\label{cap:introducao}

%[Em geral a introdução é algo que procura ser atraente, motivacional e elementar.]

De tempos pra cá, ler e ouvir falar de \defi{ciência de dados} tornou-se muito comum, tanto nos meios profissionais e científicos quanto na mídia. Existem atualmente aplicações em praticamente todas as áreas do conhecimento humano, da agricultura à indústria e ao entretenimento.

Uma busca rápida na \emph{Wikipedia} \citep{wiki} define ciência de dados como um conjunto de ferramentas que extrai informações ou previsões a partir de um grande volume de dados, que podem ser números, textos, áudio, vídeo, entre outros, para ajudar na tomada de decisões de negócios.

Apesar de não ser a única definição para o termo, Pedro A. Morettin e Julio M. Singer \citep{apostila} nos lembram que essa também é uma definição da estatística. Eles comparam o uso dos termos e apontam que o trabalho dos \emph{cientistas de dados} diferem dos \emph{estatísticos} apenas quando eles usam dados de natureza multimídia como áudio e vídeo, por exemplo. Mas que, uma vez que esses dados são processados e tornam-se números, as técnicas e conceitos utilizados pelos primeiros passam a ser basicamente os mesmos utilizados pelos segundos.

Na verdade, Morettin \& Singer \citep{apostila} citam que na década de 80 houve uma primeira tentativa de aplicar o rótulo \emph{ciência de dados}, (\emph{Data Science}), ao trabalho feito pelos estatísticos aplicados da época, como uma forma de dar-lhes mais visibilidade. Curiosamente, fato mencionado pelos autores, existem atualmente cursos específicos de ciência de dados em universidades ao redor do mundo, mas a maioria deles situada em institutos de áreas aplicadas como Engenharia e Economia, e raramente nos institutos de Estatística propriamente ditos.

Para entender um pouco mais de seu escopo, David M. Blei e Padhraic Smyth \citep{blei} discutem ciência de dados sob as visões estatística, computacional e humana. Eles argumentam que é a combinação desses três componentes que formam a essência do que ela é e, assim como, do conhecimento que ela é capaz de produzir.

Em resumo, a estatística guia a coleta e análise dos dados. A computação cria algoritmos, técnicas de processamento em paralelo e gerenciamento de memória eficazes para que sua execução seja efetiva. E o papel humano é o de avaliar quais tipos de dados, técnicas de análises, algoritmos e modelos são apropriados para responder ao problema em questão. Este é o papel do \emph{cientista de dados}.

Enquanto isso, algoritmos de \defi{aprendizado de máquina} vem sendo utilizados em grande parte dos modelos de ciência de dados. Mas o que é aprendizado de máquina? Ou então, o que significa dizer que o computador, neste caso a ``máquina'', está \emph{aprendendo}?

Antes da definição formal, Aurélien Géron \citep{hands} nos dá uma ideia geral lembrando que uma das primeiras aplicações de sucesso de aprendizado de máquina foi o filtro de \eng{spam}, criado na década de 90. Uma das fases de seu desenvolvimento foi aquela em que os usuários assinalavam que certos e-mails eram \eng{spams} e outros não eram. Hoje em dia, raramente temos que marcar ou desmarcar e-mails, pois a maioria dos filtros já ``aprenderam'' a fazer seu trabalho de forma muito eficiente.

Dentre as muitas definições de aprendizado de máquina, de modo geral uma área da ciência da computação, no contexto de ciência de dados, Joel Grus \citep{data} define-o como a ``criação e o uso de modelos que são ajustados a partir dos dados''. Seu objetivo é usar dados existentes para desenvolver modelos que possamos usar para \emph{prever} possíveis saídas para dados novos. Exemplos, além do filtro de \eng{spams} podem ser: prever transações de crédito fraudulentas, prever a chance de um cliente clicar em uma propaganda ou então prever qual time de futebol irá vencer o Campeonato Brasileiro.

Uma \defi{rede neural} é um exemplo de modelo preditivo de aprendizado de máquina. Tal modelo foi criado com inspiração no funcionamento do cérebro biológico, e que, apesar de terem sido as primeiras a serem criadas, conforme descrito por David Kopec \citep{classic}, vem ganhando nova importância na última década, graças ao avanço computacional, uma vez que exigem muito processamento, e também porque podem ser usadas para resolver problemas de aprendizagem dos mais variados tipos.

Atualmente existem vários tipos de redes neurais, porém este trabalho lida principalmente com aquele tipo que foi originalmente criado sob a inspiração do funcionamento do cérebro, chamado de \defi{perceptron}, e que portanto tenta imitar o comportamento dos neurônios e suas conexões, aprendendo padrões a partir de dados existentes e tentando prever o comportamento de dados novos a partir do padrão aprendido.

[Ao final a uma breve descrição do conteúdo do texto.]
