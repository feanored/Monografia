%!TeX root=../tese.tex
%(dica para o editor de texto: este arquivo é parte de um documento maior)
% para saber mais: https://tex.stackexchange.com/q/78101/183146

\chapter{Procedimentos de comparação e resultados}
\label{cap:comparacao}

Neste capítulo serão definidos procedimentos de modelagem alternativos às séries temporais financeiras de taxas de câmbio. Um dos modelos é o modelo paramétrico $ARIMA$, estudado no Capítulo \ref{cap:series}. 

O outro será um modelo semiparamétrico criado com redes neurais, estudadas no Capítulo \ref{cap:perceptron}, sendo utilizadas redes neurais recorrentes, devido à sua arquitetura mais condizente com o problema em questão, de acordo com Kopec \citep{classic} e Géron \citep{hands}.

A seguir, serão feitas comparações entre os modelos preditivos de séries temporais, através de medidas de desempenho comuns como percentual de acerto das previsões e observações das funções de erros dos algoritmos utilizados.

\section{Conhecendo a série temporal de interesse}

\section{Estimando um modelo ARIMA}

\section{Construção do modelo com uma rede neural}